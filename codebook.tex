\documentclass[10pt,twocolumn,oneside]{article}
\setlength{\columnsep}{10pt}                    %兩欄模式的間距
\setlength{\columnseprule}{0pt}                 %兩欄模式間格線粗細

\usepackage{amsthm}								%定義,例題
\usepackage{amssymb}
\usepackage{fontspec}							%設定字體
\usepackage{color}
\usepackage[x11names]{xcolor}
\usepackage{listings}							%顯示code用的
\usepackage{fancyhdr}							%設定頁首頁尾
\usepackage{graphicx}							%Graphic
\usepackage{enumerate}
\usepackage{titlesec}
\usepackage{amsmath}
\usepackage[CheckSingle, CJKmath]{xeCJK}
 \usepackage{CJKulem}
\usepackage{tikz}

\usepackage{amsmath, courier, listings, fancyhdr, graphicx}
\topmargin=0pt
\headsep=5pt
\textheight=740pt
\footskip=0pt
\voffset=-50pt
\textwidth=545pt
\marginparsep=0pt
\marginparwidth=0pt
\marginparpush=0pt
\oddsidemargin=0pt
\evensidemargin=0pt
\hoffset=-42pt

%\renewcommand\listfigurename{圖目錄}
%\renewcommand\listtablename{表目錄}

%%%%%%%%%%%%%%%%%%%%%%%%%%%%%

\setmainfont{Consolas}
\setmonofont{Consolas}
\setCJKmainfont{Noto Sans CJK TC}
\XeTeXlinebreaklocale "zh"						%中文自動換行
\XeTeXlinebreakskip = 0pt plus 1pt				%設定段落之間的距離
\setcounter{secnumdepth}{3}						%目錄顯示第三層

%%%%%%%%%%%%%%%%%%%%%%%%%%%%%
\makeatletter
\lst@CCPutMacro\lst@ProcessOther {"2D}{\lst@ttfamily{-{}}{-{}}}
\@empty\z@\@empty
\makeatother
\lstset{										% Code顯示
    language=C++,									% the language of the code
    basicstyle=\footnotesize\ttfamily, 					% the size of the fonts that are used for the code
    %numbers=left,									% where to put the line-numbers
    numberstyle=\footnotesize,					% the size of the fonts that are used for the line-numbers
    stepnumber=1,									% the step between two line-numbers. If it's 1, each line  will be numbered
    numbersep=5pt,									% how far the line-numbers are from the code
    backgroundcolor=\color{white},				% choose the background color. You must add \usepackage{color}
    showspaces=false,								% show spaces adding particular underscores
    showstringspaces=false,						% underline spaces within strings
    showtabs=false,								% show tabs within strings adding particular underscores
    frame=false,										% adds a frame around the code
    tabsize=2,										% sets default tabsize to 2 spaces
    captionpos=b,									% sets the caption-position to bottom
    breaklines=true,								% sets automatic line breaking
    breakatwhitespace=false,						% sets if automatic breaks should only happen at whitespace
    escapeinside={\%*}{*)},						% if you want to add a comment within your code
    morekeywords={*},								% if you want to add more keywords to the set
    keywordstyle=\bfseries\color{Blue1},
    commentstyle=\itshape\color{Red4},
    stringstyle=\itshape\color{Green4},
}


\begin{document}
\pagestyle{fancy}
\fancyfoot{}
\fancyhead[C]{National Chiao Tung University}
\fancyhead[L]{NCTU\_Kemono }
\fancyhead[R]{(\today) Page \thepage}
\renewcommand{\headrulewidth}{0.4pt}
\renewcommand{\contentsname}{Contents}

\scriptsize
\tableofcontents
\newpage
\section{Basic}
    \subsection{vimrc}
        \lstinputlisting [language=bash] {Basic/vimrc}

\section{Flow}
    \subsection{Dinic}
        \lstinputlisting [language=c++] {Flow/Dinic.cpp}
    \subsection{MCMF}
        \lstinputlisting [language=c++] {Flow/MCMF.cpp}

\section{DataStructure}
    \subsection{BIT}
        \lstinputlisting [language=c++] {DataStructure/BIT.cpp}
    \subsection{DisjointSet}
        \lstinputlisting [language=c++] {DataStructure/djs.cpp}
    \subsection{HeavyLightDecomposition}
        \lstinputlisting [language=c++] {DataStructure/HeavyLightDecomposition.cpp}
    \subsection{LCA}
        \lstinputlisting [language=c++] {DataStructure/LCA.cpp}
    \subsection{MO}
        \lstinputlisting [language=c++] {DataStructure/MO.cpp}
    \subsection{PartitionTree}
        \lstinputlisting [language=c++] {DataStructure/PartitionTree.cpp}
    \subsection{PersistentSegmentTree}
        \lstinputlisting [language=c++] {DataStructure/PersistentSegmentTree.cpp}
    \subsection{PersistentTreap}
        \lstinputlisting [language=c++] {DataStructure/PersistentTreap.cpp}
    \subsection{SparseTable}
        \lstinputlisting [language=c++] {DataStructure/SparseTable.cpp}

\section{Geometry}
    \subsection{CH}
        \lstinputlisting [language=c++] {Geometry/CH.cpp}
    \subsection{ClosestPair}
        \lstinputlisting [language=c++] {Geometry/ClosestPair.cpp}
    \subsection{HalfPlaneInter}
        \lstinputlisting [language=c++] {Geometry/HalfPlaneInter.cpp}
    \subsection{Point}
        \lstinputlisting [language=c++] {Geometry/Point.cpp}
    \subsection{PointInPoly}
        \lstinputlisting [language=c++] {Geometry/PointInPoly.cpp}
    \subsection{RotatingCaliper}
        \lstinputlisting [language=c++] {Geometry/RotatingCaliper.cpp}
    \subsection{SSinter}
        \lstinputlisting [language=c++] {Geometry/SSinter.cpp}

\section{Graph}
    \subsection{BCC}
        \lstinputlisting [language=c++] {Graph/BCC.cpp}
    \subsection{Blossom}
        \lstinputlisting [language=c++] {Graph/Blossom.cpp}
    \subsection{CutBridge}
        \lstinputlisting [language=c++] {Graph/CutBridge.cpp}
    \subsection{Dijkstra}
        \lstinputlisting [language=c++] {Graph/Dijkstra.cpp}
    \subsection{MaximumClique}
        \lstinputlisting [language=c++] {Graph/MaximumClique.cpp}
    \subsection{MinMeanCycle}
        \lstinputlisting [language=c++] {Graph/MMC.cpp}
    \subsection{SCC}
        \lstinputlisting [language=c++] {Graph/SCC.cpp}
    \subsection{TreeDiameter}
        \lstinputlisting [language=c++] {Graph/TreeDiameter.cpp}

\section{Math}
    \subsection{bigN}
        \lstinputlisting [language=c++] {Math/bigN.cpp}
    \subsection{BSGS}
        \lstinputlisting [language=c++] {Math/BSGS.cpp}
    \subsection{CRT}
        \lstinputlisting [language=c++] {Math/CRT.cpp}
    \subsection{ExtgcdModInv}
        \lstinputlisting [language=c++] {Math/ExtgcdModInv.cpp}
    \subsection{FFT}
        \lstinputlisting [language=c++] {Math/FFT.cpp}
    \subsection{Karatsuba}
        \lstinputlisting [language=c++] {Math/Karatsuba.cpp}
    \subsection{Matrix}
        \lstinputlisting [language=c++] {Math/Matrix.cpp}
    \subsection{LinearPrime}
        \lstinputlisting [language=c++] {Math/LinearPrime.cpp}
    \subsection{MU}
        \lstinputlisting [language=c++] {Math/Mobius.cpp}
    \subsection{MillerRabin}
        \lstinputlisting [language=c++] {Math/MillerRabin.cpp}

\section{String}
    \subsection{ACAutomaton}
        \lstinputlisting [language=c++] {String/ACAutomaton.cpp}
    \subsection{Eertree}
        \lstinputlisting [language=c++] {String/Eertree.cpp}
    \subsection{KMP}
        \lstinputlisting [language=c++] {String/KMP.cpp}
    \subsection{minRotation}
        \lstinputlisting [language=c++] {String/minRotation.cpp}
    \subsection{SAM}
        \lstinputlisting [language=c++] {String/SAM.cpp}
    \subsection{Z}
        \lstinputlisting [language=c++] {String/Z.cpp}


\end{document}
